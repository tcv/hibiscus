\documentclass[9pt]{article}
\usepackage{graphicx}

\begin{document}

\title{HIbiscus Data Calibration}
\maketitle

\section{Noise Circuit}
To perform data analysis of the system I wrote down a circuit layout for the noise of the first stage amplifier. I assumed that there were two contributions to the noise. First, a voltage noise source independent of the input impedence to the amplifier. Second, a current noise source dependent on the input impedence to the amplifier. \\

Given this model you get equations \ref{eq:short}, \ref{eq:50ohm}, \ref{eq:100ohm}, \ref{eq:noise} and \ref{eq:ant} for the measured voltage through the system for the different data collection modes.

\begin{center}
\begin{equation}\label{eq:short}
V^{sh}_{meas} = \mathbf{G} V_n
\end{equation}

\begin{equation} \label{eq:50ohm}
V^{50 \Omega}_{meas} = \mathbf{G}\left( \frac{Z_aZ_{50}I_n}{Z_a+Z_{50}}+\frac{Z_a (V_{J50}+V_n)}{Z_a+Z_{50}} \right)
\end{equation}

\begin{equation} \label{eq:100ohm}
V^{100 \Omega}_{meas} = \mathbf{G}\left( \frac{Z_aZ_{50}I_n}{Z_a+Z_{50}}+\frac{Z_a (V_{J100}+V_n)}{Z_a+Z_{100}} \right)
\end{equation}

\begin{equation} \label{eq:noise}
V^{noise}_{meas} = \mathbf{G}\left( \frac{Z_aZ_{50}I_n}{Z_a+Z_{50}}+\frac{Z_a (V_{J50}+V_n+V_{ns})}{Z_a+Z_{50}} \right)
\end{equation}

\begin{equation} \label{eq:ant}
V^{antenna}_{meas} = \mathbf{G}\left( \frac{Z_aZ_{ant}I_n}{Z_a+Z_{ant}}+\frac{Z_a(V_n+V_{sky})}{Z_a+Z_{ant}} \right)
\end{equation}
\end{center}

The impedences in the above equations can be measured as a function of frequency (f) using a Vector Network Analyzer (VNA). Our measurements give $Z_{50} = 50 \Omega$, while $Z_{100} = 100 \Omega * e^{\mathbf{i} 2 \pi f t_d}$, as there is a measured time delay ($t_d$) of 400 ps for the $100 \Omega$ terminator. The antenna impedence ($Z_{ant}$) and amplifier impedence ($Z_a$) were also measured usingthe VNA, but there may be an additional time delay between the two that were unable to measure directly.\\

The other known variables in the equations are the Johnson-Nyquist Noise terms ($V_{J50}$ and $V_{J100}$), which have values determined by equation \ref{eq:vj}, with $T_{amb}$ being the temperature at that location in units of Kelvin. 

\begin{center}
\begin{equation} \label{eq:vj}
V_{JZ} = \sqrt{4kT_{amb}|Z| \Delta f}
\end{equation}
\end{center}

Using equations \ref{eq:short}, \ref{eq:50ohm} and \ref{eq:100ohm}, we can solve for the Gain, Voltage Noise, and Current Noise terms ($\mathbf{G}$, $V_n$, and $I_n$). The details of this derivation are shown below. \\

First we define $\Delta V^{50}$ and $\Delta V^{100}$. 

\begin{center}
\begin{equation} \label{eq:dV}
\Delta V^Z = V^{Z}_{meas} - \frac{Z_a}{Z_a+Z}V^{sh}_{meas}
\end{equation}
\end{center}

Then we use the $\Delta V^Z$ terms to get equations \ref{eq:gain}, \ref{eq:Vn} and \ref{eq:In}.

\begin{center}
\begin{equation} \label{eq:gain}
\mathbf{G} = \frac{\Delta V^{50} (Z_a+Z_{50})Z_{100} - \Delta V^{100} (Z_a+Z_{100}) Z_{50}}{Z_a (V_{J50} Z_{100}-V_{J100} Z_{50})}
\end{equation}

\begin{equation} \label{eq:Vn}
V_n = \frac{V^{sh}_{meas}}{\mathbf{G}}
\end{equation}

\begin{equation} \label{eq:In}
I_n = \frac{V^{50 \Omega}_{meas} (Z_a+Z_{50})}{\mathbf{G} Z_a Z_{50}}-\frac{V_{J50}+V_n}{Z_{50}}
\end{equation}
\end{center}

The computer actually measures the power $P_{meas}$ for the short, $50 \Omega$, $100 \Omega$, noise and antenna data. This power is measured in dB, so to relate this data to our $V_{meas}$ data, we use two equations. First, we convert the data to linear scale via equation \ref{eq:dB}. Then we convert power to voltage using equation \ref{eq:power}. 

\begin{center}
\begin{equation} \label{eq:dB}
P^{lin}_{meas} = 10^{P^{dB}_{meas}/10}
\end{equation}

\begin{equation} \label{eq:power}
P_{meas} = \frac{\mathbf{G}V_{meas}^2 \Re (Z)}{2 |Z|^2}
\end{equation}
\end{center}


Equations \ref{eq:gain}, \ref{eq:Vn} and \ref{eq:In} are used to derive the values of $\mathbf{G}(f)$, $V_n(f)$ and $I_n(f)$ for each set of calibrator datasets. The average is taken over the battery life cycle and smoothed to remove any thermal noise left in the functions. \\

Using these functions, we can then get $V_{sky}$ from equation \ref{eq:ant}. Our final stage is to convert the voltages into temperatures. This conversion can be done by using the fact that $T$ is proportional to $V^2$. Therefore, $T_{sky}/{V_{sky}^2}$ should be the same as $T_{amb}/V_{amb}^2$. \\

In our calculation we used $T_{amb} = 300 K$ and $V_{amb}^2 = V_{J50}^2 Z_a/(Z_a+Z_{50})$. The final equation (\ref{eq:Tsky}) is used to give us a set of sky temperatures that can be compared to the GSM data. 

\begin{center}
\begin{equation} \label{eq:Tsky}
T_{sky} = \frac{T_{amb}}{V_{amb}^2} V_{sky}^2
\end{equation}
\end{center}

\section{Results}


\end{document}